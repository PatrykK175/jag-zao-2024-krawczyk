\documentclass{article}
\renewcommand{\thesubsection}{\thesection.\alph{subsection}}
\usepackage{titlesec}
\titleformat{\subsection} {\normalfont\fontsize{14}{17}\selectfont}{\thesubsection}{1em}{}
\usepackage{graphicx} % Required for inserting images
\usepackage[T1]{fontenc}
\usepackage{amsmath, amssymb}

\title{jag-zao-z01}
\author{Patryk Krawczyk}
\date{March 2024}

\begin{document}

\maketitle

\section{Zbadaj ponizsze formalne przedstawienia zbiorów, aby zrozumieć, jakie elementy zawierają. \newline Dla kazdego zbioru napisz krótki nieformalny opis w języku potocznym}
\subsection{\{1,3,5,7,...\} 
\newline Zbiór składa się z liczb nieparzystych należących do zbioru liczb naturalnych }
\subsection{\{...,-4,-2,0,2,4,...\} 
\newline Zbiór składa się z liczb parzystych należących do zbioru liczb całkowitych }
\subsection{\{n | n = 2 * m dla pewnego m należącego do $\mathbb{N}$ \}
\newline Zbiór składa się z liczb parzystych} 
\subsection{\{n | n = 2 * m dla pewnego m należącego do $\mathbb{N}$, oraz n = 3 * k dla pewnego k z $\mathbb{N}$\}
\newline Zbiór składa się z liczb naturalnych podzielnych przez liczbę 2 oraz 3} 
\subsection{\{w | w jest słowem złożonym z zer i jedynek i w jest równe odwróceniu w\}
\newline Zbiór składa się z liczb naturalnych w zakresie 0 - 1}
\subsection{\{n | n jest liczbą całkowitą i n = n + 1\}
\newline Jest to zbiór pusty}

\newpage
\section{Zapisz w formalny sposób definicje następujących zbiorów}
\subsection{zbiór zawierający liczby 1, 10, 100 - \{1,10,100\}} 
\subsection{zbiór zawierający wszystkie liczby całkowite większe niż 5 - \{n $\in$ $\mathbb{Z}$ | n > 5\}} 
\subsection{zbiór zawierający wszystkie liczby naturalne mniejsze niż 5 - \{n $\in$ $\mathbb{N}$ | n < 5\}} 
\subsection{zbiór zawierający słowo aba - \{"aba"\}}
\subsection{zbiór zawierający słowo puste - \{\( \varepsilon \)\}}
\subsection{zbiór niezawierający zupełnie nic - \( \emptyset \)}
\end{document}
